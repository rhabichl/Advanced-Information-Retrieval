\documentclass[12pt,a4paper]{article}

% ----- PACKAGES -----
\usepackage[utf8]{inputenc}
\usepackage{graphicx}
\usepackage{geometry}
\usepackage{setspace}
\usepackage{hyperref}
\usepackage{fancyhdr}
\usepackage{titlesec}
\usepackage[maxbibnames=3,minbibnames=1,maxcitenames=2,mincitenames=1,style=numeric-comp,backend=biber]{biblatex}
\addbibresource{references.bib}


\geometry{margin=1in}
\setstretch{1.2}
\pagestyle{fancy}
\fancyhf{}
\rhead{Project Design Document}
\lhead{\leftmark}
\rfoot{Page \thepage}

\title{
    \vspace{2cm}
    \textbf{Advanced Information Retrieval}\\[0.5cm]
    \large Fine-Tuning and Transferability in Legal Information Retrieval\\[1.5cm]
}
\author{
    Raphael Habichler\thanks{student-id: 12419578} \and
    Mark Sesko\thanks{student-id: 12114879} \and
    Paul Brandstätter\thanks{student-id: 12212566}
}
\date{\today}

\titleformat{\section}{\large\bfseries}{\thesection}{1em}{}
\titleformat{\subsection}{\normalsize\bfseries}{\thesubsection}{1em}{}

\begin{document}
\maketitle
\section{Abstract}
Research by Hoffmann et al. \cite{hoffmann2022trainingcomputeoptimallargelanguage} suggests that data quality and quantity are among the most critical factors in training large language models (LLMs), which, at the time of writing this design document, dominate the rankings in text embedding benchmarks such as MTEB \cite{MTEB} and MMTEB \cite{MMTEB}.

To mitigate the need for massive amounts of training data, we employed a state-of-the-art open-weight embedding model, Qwen3-Embedding-8B \cite{qwen3embedding}, which we fine-tuned on pre-labeled German domain-specific datasets \cite{wrzalik-krechel-2021-gerdalir}. We then evaluated its retrieval performance across multiple jurisdictions, specifically German \cite{9723721} and Chinese \cite{li2023lecardv2} legal corpora.

This approach enables users to retrieve semantically similar documents across jurisdictions, significantly facilitating international legal communication and comparative analysis.

\section{Methodology}
Figure~\ref{fig:flowchart} shows our research workflow. We collect legal documents from Austrian, German, and Chinese sources, then prepare the data for model training and evaluation.

We test two approaches: first, using the pre-trained Qwen3-Embedding-8B model directly on all three datasets. Second, fine-tuning the model on German legal data using cosine similarity between queries and documents. We then evaluate both models using Precision@k, Recall@k, and nDCG@k metrics to compare their performance across jurisdictions.

\begin{figure}[h]
\centering
\includegraphics[width=\textwidth]{flowchart.pdf}
\caption{Project workflow and methodology}
\label{fig:flowchart}
\end{figure}

\newpage
\printbibliography 
\end{document}
